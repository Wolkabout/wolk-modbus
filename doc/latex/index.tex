
\begin{DoxyCode}
██╗    ██╗ ██████╗ ██╗     ██╗  ██╗     ███╗   ███╗ ██████╗ ██████╗ ██████╗ ██╗   ██╗███████╗
██║    ██║██╔═══██╗██║     ██║ ██╔╝     ████╗ ████║██╔═══██╗██╔══██╗██╔══██╗██║   ██║██╔════╝
██║ █╗ ██║██║   ██║██║     █████╔╝█████╗██╔████╔██║██║   ██║██║  ██║██████╔╝██║   ██║███████╗
██║███╗██║██║   ██║██║     ██╔═██╗╚════╝██║╚██╔╝██║██║   ██║██║  ██║██╔══██╗██║   ██║╚════██║
╚███╔███╔╝╚██████╔╝███████╗██║  ██╗     ██║ ╚═╝ ██║╚██████╔╝██████╔╝██████╔╝╚██████╔╝███████║
 ╚══╝╚══╝  ╚═════╝ ╚══════╝╚═╝  ╚═╝     ╚═╝     ╚═╝ ╚═════╝ ╚═════╝ ╚═════╝  ╚═════╝ ╚══════╝
\end{DoxyCode}


This is Modbus R\+TU slave library written in C99. It is intended to be used on a bare-\/metal device. Hardware communication interface needs to be implemented by the user. The usual communication interface is U\+A\+R\+T/\+U\+S\+A\+RT. Modbus itself needs to be configured by setting the number and type of wanted registers. Supported Modbus registers are\+:
\begin{DoxyItemize}
\item Holding Register
\item Input Register
\item Coils
\item Input Bits
\end{DoxyItemize}

\section*{Usage}

{\ttfamily \mbox{\hyperlink{driver__init_8h}{driver\+\_\+init.\+h}}} lists and describes interfaces that user needs to provide for establishing communication. This file needs to be included in user space. Three hardware interfaces are needed from M\+CU\+:
\begin{DoxyItemize}
\item timer -\/ with counter(\+I\+R\+Q) on every 1ms
\item U\+A\+R\+T/\+U\+S\+A\+RT -\/ receiving bytes one by one. It is recommended to use both interrupts -\/ for Tx and Rx.
\item Two signal diodes(optional) -\/ used to visualize bytes received and bytes transmited.
\end{DoxyItemize}

{\ttfamily \mbox{\hyperlink{modbus__configuration_8h}{modbus\+\_\+configuration.\+h}}} requires setup of Modbus interfaces. The following needs to be defined\+:
\begin{DoxyItemize}
\item Number of registers -\/ separatly for holding registers, input registers, coils and input bits. Eg\+: {\ttfamily \#define H\+O\+L\+D\+I\+N\+G\+\_\+\+R\+E\+G\+I\+S\+T\+ER 1}
\item A unique name for each position in each register type. If you define 3 holding registers then you will define 3 different names in enumeration. Eg\+: {\ttfamily H\+O\+L\+D\+I\+N\+G\+\_\+\+R\+E\+G\+I\+S\+T\+E\+R\+\_\+1}
\item For each register type, define unique address in struct \mbox{\hyperlink{structmodbus__variables__map__t}{modbus\+\_\+variables\+\_\+map\+\_\+t}}. Eg\+: {\ttfamily modbus\+\_\+variable\+\_\+t holding\+\_\+register\+\_\+1;}
\item Map all defined registers in \mbox{\hyperlink{group___insert_ga6dbaaf108ca00d5dfd1037aadfc10f6f}{modbus\+\_\+register\+\_\+mapping\+\_\+init()}}. Eg\+: {\ttfamily modbus\+\_\+variable\+\_\+init(\&modbus\+\_\+variables\+\_\+map.\+holding\+\_\+register\+\_\+1, \&Holding\+Register\mbox{[}H\+O\+L\+D\+I\+N\+G\+\_\+\+R\+E\+G\+I\+S\+T\+E\+R\+\_\+1\mbox{]});}
\end{DoxyItemize}

\paragraph*{User space}

These A\+PI functions need to be called before starting the manipulation of Modbus registers\+: 
\begin{DoxyCode}
//init all HW interfaces
rs485\_interface\_init();

//Init Modbus, set slave address
modbus\_init(SLAVE\_ADDRESS);

//Init your own reigsters
modbus\_register\_mapping\_init();
\end{DoxyCode}


To set the value of any modbus reigster use\+: {\ttfamily modbus\+\_\+variable\+\_\+set\+\_\+value()}

To read the current value of any register check the register\textquotesingle{}s array by navigating to wanted position\+:
\begin{DoxyItemize}
\item Holding\+Register\mbox{[}H\+O\+L\+D\+I\+N\+G\+\_\+\+R\+E\+G\+I\+S\+T\+ER\mbox{]}
\item Input\+Register\mbox{[}I\+N\+P\+U\+T\+\_\+\+R\+E\+G\+I\+S\+T\+ER\mbox{]}
\item Coils\mbox{[}C\+O\+I\+LS\mbox{]}
\item Input\+Bits\mbox{[}I\+N\+P\+U\+T\+\_\+\+B\+I\+TS\mbox{]} 
\end{DoxyItemize}